%%%%%%%%%%%%%%% Stochastic Processes (Random Walks, Martingales, Brownian Motion)

\section{Stochastic Processes}

These notes are based on my notes from \textit{Time Series and Panel Data Econometrics} (1st edition) by M. Hashem Pesaran as well as coursework for Economics 613: Economic and Financial Time Series I at USC.

\subsection{Martingales}

\textbf{Definition.} Let \(\{y_t\}_{t=0}^\infty \) be a sequence of random variables, and let \(\Omega_t\) denote the information set available at date \(t\), which at least contains \(\{y_t, y_{t-1}, y_{t-2}, \ldots \}\). If \(\E(y_t \mid \Omega_{t-1}) = y_{t-1}\) holds then \(\{y_t\}\) is a martingale process with respect to \(\Omega_t\).

\textbf{Definition.} Let \(\{y_t\}_{t=1}^\infty \) be a sequence of random variables, and let \(\Omega_t\) denote the information set available at date \(t\), which at least contains \(\{y_t, y_{t-1}, y_{t-2}, \ldots \}\). If \(\E(y_t \mid \Omega_{t-1}) =0\), then \(\{y_t\}\) is a martingale difference process with respect to \(\Omega_t\).

\subsection{Brownian Motion}

% \textbf{The first distribution is in appendix B.13.1, formula B.52 (p.984 of book). Supposedly the proof is in Phillips and Durlaf (1986), which is now in the Google drive folder.}
%
\textbf{Appendix B.13, Brownian motion.} A standard Brownian motion \(b(\cdot)\) is a continuous-time stochastic process associating each date \(a \in [0, 1]\) with the scalar \(b(a)\) such that

\begin{enumerate}[(i)]

\item b(0) = 0

\item For any dates \(0 \leq a_1 \leq a_2 \leq \ldots \leq a_k \leq 1\) the changes \([b(a_2) - b(a_1)]\), \([b(a_3) - b(a_2)], \ldots, [b(a_k) - b(a_k - 1)]\) are independent multivariate Gaussian with \(b(a) - b(s) \sim \mathcal{N}(0, a -s)\). 

\item For any given realization, \(b(a)\) is continuous in \(a\) with probability 1.

\end{enumerate}

Other continuous time processes can be generated from the standard Brownian motion. For example, a Brownian motion with variance \(\sigma^2\) can be obtained as

\[
w(a) = \sigma b(a)
\]

where \(b(a)\) is a standard Brownian motion.

\

The continuous time process

\[
\boldsymbol{w}(a) = \boldsymbol{\Sigma}^{1/2} \boldsymbol{b}(a)
\]

is a Brownian motion with covariance matrix \(\boldsymbol{\Sigma}\).

\textbf{Definition 26 (Wiener process).} Let \(\Delta w(t)\) be the change in \(w(t)\) during the time interval \(dt\). Then \(w(t)\) is said to follow a Wiener process if

\[
\Delta w(t) = \epsilon_t \sqrt{dt}, \ \ \epsilon_t \sim IID(0, 1)
\]

and \(w(t)\) denotes the value of the \(w(\cdot)\) at date \(t\). Clearly,

\[
\E[\Delta w(t)] = 0, \text{ and } \Var[ \Delta w(t)] = dt
\]

\textbf{Donsker's Theorem, Theorem 43, p.335, Section 15.6.3.} Let \(a \in [0, 1)\), \(t \in [0, T]\), and suppose \((J - 1)/T \leq a < J/T, J = 1, 2, \ldots, T\). Define

\[
R_T(a) = \frac{1}{\sqrt{T}} s_{ \big[Ta \big] }
\]

where

\[
s_{ \big[Ta \big] } = \epsilon_1 + \epsilon_2 + \ldots + \epsilon_{ \big[Ta \big] }
\]

\(\big[Ta \big]\) denotes the largest integer part of \(Ta\) and \(s_{ \big[Ta \big] } = 0\) if \(\big[Ta \big] = 0\). Then \(R_T(a)\) weakly converges to \(w(a)\), i.e., 

\[
R_T(a) \to w(a)
\]

where \(w(a)\) is a Wiener process. Note that when \(a = 1\), \(R_T(1) = 1/\sqrt{T} \cdot S_{\big[T \big]} = 1/\sqrt{T} \cdot (\epsilon_1 + \epsilon_2 + \ldots + \epsilon_T\). Since \(\epsilon_t\)'s are IID, by the central limit theorem, \(R_T(1) \to \mathcal{N}(0, 1)\). 

Similar (Theorem 2.1 in  Phillips and Durlaf (1986)): Let \(\{u_t\}\) be a sequence satisfying \(\E(u_t) = 0\), \( \gamma(0) = \E(T^{-1}S_t ^2) \to \sigma^2 < \infty \text{ as } T \to \infty\), \(\{u_t\}\) is square summable, \(\sup_t \{ \E( |u_t|^\beta) \} < \infty\) for some \(2 \leq \beta < \infty\) and all \(t\), \(\gamma(h) = \E(T^{-1}(y_t - y_{t-h})^2) \to K_h < \infty\) as \(\min \{h, T\} \to \infty\). Then \(X_T(t) \implies W(t)\) as \(T \to \infty\), where \(W(t)\) is a Wiener process.

%\end{enumerate}