\documentclass{article}

\usepackage{fancyhdr}
\usepackage{extramarks}
\usepackage{amsmath}
\usepackage{amsthm}
\usepackage{amsfonts}
\usepackage{tikz}
\usepackage{enumerate}
\usepackage{graphicx}
\graphicspath{ {images/} }
\usepackage[plain]{algorithm}
\usepackage{algpseudocode}
\usepackage[document]{ragged2e}
\usepackage{textcomp}
\usepackage{color}   %May be necessary if you want to color links
\usepackage{import}
\usepackage{hyperref}
\hypersetup{
    colorlinks=true, %set true if you want colored links
    linktoc=all,     %set to all if you want both sections and subsections linked
    linkcolor=black,  %choose some color if you want links to stand out
}

\usetikzlibrary{automata,positioning}


% Basic Document Settings


\topmargin=-0.45in
\evensidemargin=0in
\oddsidemargin=0in
\textwidth=6.5in
\textheight=9.0in
\headsep=0.25in
\setlength{\parskip}{1em}

\linespread{1.1}

\pagestyle{fancy}
\lhead{\hmwkAuthorName}
\lfoot{\lastxmark}
\cfoot{\thepage}

\renewcommand\headrulewidth{0.4pt}
\renewcommand\footrulewidth{0.4pt}

\setlength\parindent{0pt}


\newcommand{\hmwkTitle}{Math Review Notes---Convex Optimization}
\newcommand{\hmwkAuthorName}{\textbf{G. Faletto} }


%%%%% Title Page


\title{
    \vspace{2in}
    \textmd{\textbf{ \hmwkTitle}}\\
}

\author{Gregory Faletto}
\date{}

\renewcommand{\part}[1]{\textbf{\large Part \Alph{partCounter}}\stepcounter{partCounter}\\}


%%%%% Various Helper Commands


%%%%% Useful for algorithms
\newcommand{\alg}[1]{\textsc{\bfseries \footnotesize #1}}

%%%%% For derivatives
\newcommand{\deriv}[2]{\frac{\mathrm{d} #1}{\mathrm{d} #2}}

%%%%% For partial derivatives
\newcommand{\pderiv}[2]{\frac{\partial #1}{\partial #2}}

%%%%% Integral dx
\newcommand{\dx}{\mathrm{d}x}

%%%%% Alias for the Solution section header
\newcommand{\solution}{\textbf{\large Solution}}

%%%%% Probability commands: Expectation, Variance, Covariance, Bias
\newcommand{\E}{\mathbb{E}}
\newcommand{\Var}{\mathrm{Var}}
\newcommand{\Cov}{\mathrm{Cov}}
\newcommand{\Bias}{\mathrm{Bias}}
\newcommand\indep{\protect\mathpalette{\protect\independenT}{\perp}}
\def\independenT#1#2{\mathrel{\rlap{$#1#2$}\mkern2mu{#1#2}}}
\DeclareMathOperator{\Tr}{Tr}

\theoremstyle{definition}
\newtheorem{theorem}{Theorem}
\theoremstyle{definition}
\newtheorem{proposition}[theorem]{Proposition}
\theoremstyle{definition}
\newtheorem{lemma}[theorem]{Lemma}
\theoremstyle{definition}
\newtheorem{corollary}{Corollary}[theorem]
\theoremstyle{definition}
\newtheorem{definition}{Definition}[section]
\newtheorem*{remark}{Remark}

%%%%% Tilde
\newcommand{\textapprox}{\raisebox{0.5ex}{\texttildelow}}

\begin{document}

\maketitle

\pagebreak

%\tableofcontents

%\
%
%\
%
%\begin{center}
%Last updated \today
%\end{center}
%
%
%
%\newpage

%
%
%
%
%
%
%
%
%%
%% Convex Optimization

\section{Convex Optimization}

These are my notes from taking EE 588 at USC and the textbook \textit{Convex Optimization} (Boyd and Vandenberghe) 7th printing.

\textbf{Need to cover:}

\begin{itemize}

\item Update rules for optimization problems (e.g. gradient descent, be able to write down gradient, etc.)

\item Know which algorithms are useful in which settings

\item Homework-like problems from first part of class (no proofs though) (Boyd homework is good practice)

\item Understand how to derive algorithms

\item Understand how to calculate gradients, proximal functions, etc.

\item Understand examples, how to run algorithms

\item Only conceptual thing: duality question (write down dual)

\item Formulate problems as convex optimization problems

\end{itemize}

\textbf{Do not need to cover:}

\begin{itemize}

\item ADMM

\item Proofs from 2nd half of class (rates of convergence, etc.)

\item Coding

\end{itemize}


%
%
%
%
%
%
%
%

\end{document}