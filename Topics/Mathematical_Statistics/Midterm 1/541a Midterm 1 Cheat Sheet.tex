



\documentclass{article}

\usepackage{fancyhdr}
\usepackage{extramarks}
\usepackage{amsmath}
\usepackage{amsthm}
\usepackage{amsfonts}
\usepackage{tikz}
\usepackage{enumerate}
\usepackage{graphicx}
\graphicspath{ {images/} }
\usepackage[plain]{algorithm}
\usepackage{algpseudocode}
\usepackage[document]{ragged2e}
\usepackage{textcomp}
\usepackage{color}   %May be necessary if you want to color links
\usepackage{import}
\usepackage{hyperref}
\hypersetup{
    colorlinks=true, %set true if you want colored links
    linktoc=all,     %set to all if you want both sections and subsections linked
    linkcolor=black,  %choose some color if you want links to stand out
}

\usetikzlibrary{automata,positioning}


% Basic Document Settings


\topmargin=-0.45in
\evensidemargin=0in
\oddsidemargin=0in
\textwidth=6.5in
\textheight=9.0in
\headsep=0.25in
\setlength{\parskip}{1em}

\linespread{1.1}

\pagestyle{fancy}
\lhead{\hmwkAuthorName}
\lfoot{\lastxmark}
\cfoot{\thepage}

\renewcommand\headrulewidth{0.4pt}
\renewcommand\footrulewidth{0.4pt}

\setlength\parindent{0pt}


\newcommand{\hmwkTitle}{Math 541A Midterm 1 Cheat Sheet}
\newcommand{\hmwkAuthorName}{\textbf{G. Faletto} }


%%%%% Title Page


\title{
    \vspace{2in}
    \textmd{\textbf{ \hmwkTitle}}\\
}

\author{Gregory Faletto}
\date{}

\renewcommand{\part}[1]{\textbf{\large Part \Alph{partCounter}}\stepcounter{partCounter}\\}


%%%%% Various Helper Commands


%%%%% Useful for algorithms
\newcommand{\alg}[1]{\textsc{\bfseries \footnotesize #1}}

%%%%% For derivatives
\newcommand{\deriv}[2]{\frac{\mathrm{d} #1}{\mathrm{d} #2}}

%%%%% For partial derivatives
\newcommand{\pderiv}[2]{\frac{\partial #1}{\partial #2}}

%%%%% Integral dx
\newcommand{\dx}{\mathrm{d}x}

%%%%% Alias for the Solution section header
\newcommand{\solution}{\textbf{\large Solution}}

%%%%% Probability commands: Expectation, Variance, Covariance, Bias
\newcommand{\E}{\mathbb{E}}
\newcommand{\Var}{\mathrm{Var}}
\newcommand{\Cov}{\mathrm{Cov}}
\newcommand{\Bias}{\mathrm{Bias}}
\newcommand\indep{\protect\mathpalette{\protect\independenT}{\perp}}
\def\independenT#1#2{\mathrel{\rlap{$#1#2$}\mkern2mu{#1#2}}}
\DeclareMathOperator{\Tr}{Tr}

\theoremstyle{definition}
\newtheorem{theorem}{Theorem}
\theoremstyle{definition}
\newtheorem{proposition}[theorem]{Proposition}
\theoremstyle{definition}
\newtheorem{lemma}[theorem]{Lemma}
\theoremstyle{definition}
\newtheorem{corollary}{Corollary}[theorem]
\theoremstyle{definition}
\newtheorem{definition}{Definition}[section]
\newtheorem*{remark}{Remark}
\theoremstyle{definition}
\newtheorem{exercise}{Exercise}
\theoremstyle{definition}
\newtheorem{example}{Example}[section]

%%%%% Tilde
\newcommand{\textapprox}{\raisebox{0.5ex}{\texttildelow}}

\begin{document}

\maketitle

\pagebreak

%\tableofcontents

%\
%
%\
%
%\begin{center}
%Last updated \today
%\end{center}



\newpage
%
%%
%%
%%
%%
%%
%%
%%
%%
%%%
%%% Probability


\section{Review of Probability Theory}

\begin{proposition}[\textbf{Change of Variables}]If \(U\) is a ``nice" subset of \(\mathbb{R}^2\) and \(\phi\) is an injective differentiable function on \(U\), then 

\[
 \int_{\phi(U)} f(u,v) dudv = \int_U f(\phi(x,y)) | J \phi(x,y) | dx dy
\]

where \( J \phi(x,y)\) is the Jacobian of \(\phi\) at \((x,y)\).

\end{proposition}

\begin{definition}Random variables \(X_1, X_2, \ldots, X_n\) are independent if for every \(B_1, B_2, \ldots, B_n \subseteq \mathbb{R}\), the events \(\{X_1 \in B_1\}, \{X_2 \in B_2\}, \ldots, \{X_n \in B_n\} \) are independent; that is, 

\[
\mathbb{P} \bigg( \bigcap_{i=1}^n \{X_i \in B_i\} \bigg) = \prod_{i=1}^n \mathbb{P}(\{X_i \in B_i\})
\]

\end{definition}

\section{Limit Theorems}

\begin{definition} \textbf{Convergence in probability.} \(\{X_n\}\) is said to \textbf{converge in probability} to \(X\) if
\begin{itemize}

\item Grimmett and Strizaker definition:
\[
\lim_{n \to \infty} \Pr(|X_n -X| > \epsilon) = 0, \text{ for every } \epsilon > 0
\]

\item More formal (from Math 541A):

\[
\forall \epsilon > 0, \ \lim_{n \to \infty} \Pr(\{\omega \in \Omega : |X_n(\omega) - X(\omega)| > \epsilon) = 0
\]

\end{itemize}

\end{definition} 

%%%% Convergence with probability 1 or almost surely
\begin{definition}
\textbf{Convergence with probability 1 or almost surely.} The sequence of random variables \(\{X_n\}\) is said to \textbf{converge with probability 1} (or \textbf{almost surely}) to \(X\) if 



\[
\Pr\big( \{\omega \in \Omega: \lim_{n \to \infty} X_n(\omega) = X(\omega) \} \big) = 1
\]



\begin{remark}This is often written as \(X_n \xrightarrow{w.p.1} X\) or \(X_n \xrightarrow{a.s.} X\). An equivalent condition for convergence with probability 1 is given by

\[
\lim_{n \to \infty} \Pr( |X_m - X| < \epsilon, \text{ for all } m \geq =n) = 1, \text{ for every } \epsilon > 0
\]

which shows that convergence in probability is a special case of convergence with probability 1 (obtained by setting \(m = n\)). Convergence with probability 1 is stronger than convergence in probability and is often referred to as ``strong convergence." \end{remark}

\end{definition}

%%%% Convergence in r-th mean
\begin{definition}
\textbf{Convergence in \(r\)-th mean} or \textbf{convergence in \(\ell_p\)}. \(X_n \to X\) \textbf{in \(r\)th mean}  (or \textbf{in \(\ell_p\)}) where \(r \geq 1\) (or \(0 < p \leq \infty\)) if \(\E|X_n^r| < \infty\) for all \(n\) and

\[
\lim_{n \to \infty} \E(|X_n - X|^r) = 0
\]

or if \(\lVert X \rVert_p < \infty\) and

\[
\lim_{n \to \infty} \lVert X_n - X \rVert _p = 0
\]

\begin{remark}
Recall that \(\lVert X \rVert_p := (\E(X)^p)^{1/p}\) if \(0 < p < \infty\) and \(\lVert X \rVert_\infty := \inf \{c >0: \Pr(|X| \leq c ) = 1\}\). Note that if \(p < 1\), \(\lVert \cdot \rVert_p\) is no longer a norm because it does not satisfy the Triangle Inequality (Corollary \ref{asym.tri.ineq.exp} and Theorem \ref{asym.tri.ineq.norm}), but this property still holds. Convergence in \(r\)th mean is often written \(X_n \xrightarrow{r} X\).
\end{remark}
\end{definition}

%%%%%%% Convergence in Distribution
\begin{definition}
\textbf{Convergence in Distribution.} Let \(X_1, X_2, \ldots\) have distribution functions \(F_1(\cdot), F_2(\cdot), \ldots \) respectively. Then \(X_n\) is said to \textbf{converge in distribution to \(X\)} if

\[
\lim_{n \to \infty} \Pr(X_n \leq u) = \Pr(X \leq u)
\]

for all \(u\) at which \(F_X(x) = \Pr(X \leq x)\) is continuous. 


\end{definition}



\section{Exponential Families}

\begin{definition}\label{prob.defn.exp.fams} Let \(n, k\) be positive integers and let \(\mu\) be a \textit{measure} on \(\mathbb{R}^n\) (that is, a probability law that does not necessarily sum to 1). Let \(t_1, \ldots, t_k : \mathbb{R}^n \to \mathbb{R}\). Let \(h: \mathbb{R}^n \to [0, \infty]\), and assume \(h\) is not identically zero. For any \(w = (w_1, \ldots, w_k) \in \mathbb{R}^k\), define

\[
a(w) := \log \bigg[ \int_{\mathbb{R}^n} h(x) \exp \bigg( \sum_{i=1}^k w_i t_i(x) \bigg) d \mu(x) \bigg], \ \ \ \forall x \in \mathbb{R}^n
\]

The set \(\{w \in \mathbb{R}^k\}\) is called the \textbf{natural parameter space}. On this set, the function

\[
f_w(x) := h(x) \exp \bigg( \sum_{i=1}^k w_i t_i(x) - a(w) \bigg), \ \ \ \forall x \in \mathbb{R}^n
\]

satisfies \(\int_{\mathbb{R}^n} f_w(x) d \mu(s) = 1\) (by the definition of \(a(w)\)). So, the set of functions (which can be interpreted as probability density functions, or as probability mass functions according to \(\mu\)) \(\{f_w: \theta \in \Theta: a(w(\theta)) < \infty \}\) is called a \textbf{\(k\)-parameter exponential family in canonical form.} 
 
 \
 
 More generally, let \(\Theta \in \mathbb{R}^k\) be any set an let \(w: \Theta \to \mathbb{R}^k\). We define a \textbf{\(k\)-parameter exponential family} to be a set of functions \(\{f_{\theta}: \theta \in \Theta\}\), where
 
 \[
 f_{\theta}(x) := h(x) \exp \bigg( \sum_{i=1}^k w_i(\theta) t_i(x) - a(w(\theta))\bigg), \ \ \ \ \forall x \in \mathbb{R}^n
 \]
 

\end{definition}


\begin{theorem}[\textbf{Theorem 3.4.2 from Casella and Berger}] If \(X\) is a random variable in an exponential family, then

\begin{equation}\label{prob.thm.3.4.2.casella}
\E \Bigg( \sum_{i=1}^k \pderiv{w_i(\theta)}{\theta_j} t_i(X) \Bigg) =  \pderiv{}{\theta_j}  a(w(\theta)).
\end{equation}

\end{theorem}


\section{Random Samples}

\subsubsection{The Delta Method}

\begin{theorem}[\textbf{Delta Method, Theorem 4.14 in 541A notes, 5.5.24 in Casella and Berger}]\label{mathstats.delta.method.thm} Let \(\theta \in \mathbb{R}\). Let \(Y_1, Y_2, \ldots\) be random variables such that \(\sqrt{n}(Y_n - \theta)\) converges in distribution to a mean zero Gaussian random variable with variance \(\sigma^2 > 0\). Let \(f: \mathbb{R} \to \mathbb{R}\). Assume that \(f'\) exists and is continuous, and \(f'(\theta) \neq 0\). Then

\[
\sqrt{n}(f(Y_n) - f(\theta)) \xrightarrow{d} \mathcal{N}(0, \sigma^2(f'(\theta))^2).
\]

%converges in distribution to a mean zero Gaussian random variable with variance \(\sigma^2(f'(\theta))^2\) as \(n \to \infty\).

\end{theorem}

\begin{theorem}[\textbf{Second Order Delta Method, Theorem 4.17 in Math 541A Notes.}]Let \(\theta \in \mathbb{R}\). Let \(Y_1, Y_2, \ldots\) be random variables such that \(\sqrt{n}(Y_n - \theta)\) converges in distribution to a mean zero Gaussian random variable with variance \(\sigma^2 > 0\). Let \(f: \mathbb{R} \to \mathbb{R}\). Assume that \(f''\) exists and is continuous, \(f'(\theta) = 0\) and \(f''(\theta) \neq 0\). Then

\[
n(f(Y_n) - f(\theta)) \xrightarrow{d} \sigma^2 \frac{1}{2} | f''(\theta)| \cdot \chi_1^2.
\]

%converges in distribution to a \(\chi_1^2\) random variable multiplied by \(\sigma^2 \frac{1}{2} | f''(\theta)|\) as \(n \to \infty\).

\end{theorem}









\end{document}