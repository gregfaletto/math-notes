\documentclass{article}

\usepackage{fancyhdr}
\usepackage{extramarks}
\usepackage{amsmath}
\usepackage{amsthm}
\usepackage{amsfonts}
\usepackage{tikz}
\usepackage{enumerate}
\usepackage{graphicx}
\graphicspath{ {images/} }
\usepackage[plain]{algorithm}
\usepackage{algpseudocode}
\usepackage[document]{ragged2e}
\usepackage{textcomp}
% \usepackage{amssymb}
\usepackage{import}
\usepackage{natbib}

\usetikzlibrary{automata,positioning}

%
% Basic Document Settings
%

\topmargin=-0.45in
\evensidemargin=0in
\oddsidemargin=0in
\textwidth=6.5in
\textheight=9.0in
\headsep=0.25in

\linespread{1.1}

\pagestyle{fancy}
\lhead{\hmwkAuthorName}
\chead{\hmwkClass\: \hmwkTitle}
%\rhead{\firstxmark}
%\lfoot{\lastxmark}
\cfoot{\thepage}

\renewcommand\headrulewidth{0.4pt}
\renewcommand\footrulewidth{0.4pt}

\setlength\parindent{0pt}

%
% Create Problem Sections
%

%\newcommand{\enterProblemHeader}[1]{
%    \nobreak\extramarks{}{Question \arabic{#1} continued on next page\ldots}\nobreak{}
%    \nobreak\extramarks{Question \arabic{#1} (cont.)}{Question \arabic{#1} continued on next page\ldots}\nobreak{}
%}
%
%\newcommand{\exitProblemHeader}[1]{
%    \nobreak\extramarks{Question \arabic{#1} (cont.)}{Question \arabic{#1} continued on next page\ldots}\nobreak{}
%    \stepcounter{#1}
%    \nobreak\extramarks{Question \arabic{#1}}{}\nobreak{}
%}
%
%\setcounter{secnumdepth}{0}
%\newcounter{partCounter}
%\newcounter{homeworkProblemCounter}
%\setcounter{homeworkProblemCounter}{1}
%\nobreak\extramarks{Problem \arabic{homeworkProblemCounter}}{}\nobreak{}

%%
%% Homework Problem Environment
%%
%% This environment takes an optional argument. When given, it will adjust the
%% problem counter. This is useful for when the problems given for your
%% assignment aren't sequential. See the last 3 problems of this template for an
%% example.
%%
%\newenvironment{homeworkProblem}[1][-1]{
%    \ifnum#1>0
%        \setcounter{homeworkProblemCounter}{#1}
%    \fi
%    \section{Problem \arabic{homeworkProblemCounter}}
%    \setcounter{partCounter}{1}
%    \enterProblemHeader{homeworkProblemCounter}
%}{
%    \exitProblemHeader{homeworkProblemCounter}
%}

\theoremstyle{definition}
\newtheorem{exercise}{Exercise}
\newtheorem{definition}{Definition}
\newtheorem*{remark}{Remark}
\theoremstyle{definition}
\newtheorem{theorem}{Theorem}
\theoremstyle{definition}
\newtheorem{lemma}[theorem]{Lemma}
\theoremstyle{definition}
\newtheorem{corollary}{Corollary}[theorem]

%
% Homework Details
%   - Title
%   - Due date
%   - Class
%   - Section/Time
%   - Instructor
%   - Author
%

\newcommand{\hmwkTitle}{2016 In-Class Exam}
%\newcommand{\hmwkDueDate}{Apr. 26, 2019}
\newcommand{\hmwkClass}{DSO}
%\newcommand{\hmwkClassTime}{Section A}
%\newcommand{\hmwkClassInstructor}{S. Heilman}
\newcommand{\hmwkAuthorName}{\textbf{G. Faletto} }

%\renewcommand{\subset}{\subseteq}
\renewcommand{\supset}{\supseteq}
\renewcommand{\epsilon}{\varepsilon}

\newcommand{\abs}[1]{\left|#1\right|}                   % Absolute value notation
\newcommand{\absf}[1]{|#1|}                             % small absolute value signs
\newcommand{\vnorm}[1]{\left|\left|#1\right|\right|}    % norm notation
\newcommand{\vnormf}[1]{||#1||}                         % norm notation, forced to be small
\newcommand{\im}[1]{\mbox{im}#1}                        % Pieces of English for math mode
\newcommand{\tr}[1]{\mbox{tr}#1}
\newcommand{\Proj}[1]{\mbox{Proj}#1}
\newcommand{\Vol}[1]{\mbox{Vol}#1}
\newcommand{\Z}{\mathbb{Z}}                             % Blackboard notation
\newcommand{\N}{\mathbb{N}}
\newcommand{\E}{\mathbb{E}}
\renewcommand{\P}{\mathbb{P}}
\newcommand{\R}{\mathbb{R}}
\newcommand{\C}{\mathbb{C}}
\newcommand{\Q}{\mathbb{Q}}
\newcommand{\figoneawidth}{.5\textwidth}                % Image formatting parameters
\newcommand{\lbreak}{\\}                                % Linebreak
\newcommand{\italicize}[1]{\textit {#1}}                % formatting commands for bibliography
%\newcommand{\embolden}[1]{\textbf {#1}}
\newcommand{\embolden}[1]{{#1}}
\newcommand{\undline}[1]{\underline {#1}}
\newcommand{\e}{\varepsilon}
\renewcommand{\epsilon}{\varepsilon}
%\renewcommand{:=}{=}

%
% Title Page
%

\title{
    \vspace{2in}
    \textmd{\textbf{DSO Screening Exam:\ \hmwkTitle}}\\
%    \normalsize\vspace{0.1in}\small{Due\ on\ \hmwkDueDate}\\
%    \vspace{0.1in}\large{\textit{Instructor: Dr. Steven Heilman\ }}
    \vspace{3in}
}

\author{Gregory Faletto}
\date{}

\renewcommand{\part}[1]{\textbf{\large Part \Alph{partCounter}}\stepcounter{partCounter}\\}

%
% Various Helper Commands
%

% Useful for algorithms
\newcommand{\alg}[1]{\textsc{\bfseries \footnotesize #1}}

% For derivatives
\newcommand{\deriv}[2]{\frac{\mathrm{d} #1}{\mathrm{d} #2}}

% For partial derivatives
\newcommand{\pderiv}[2]{\frac{\partial #1}{\partial #2}}

% Integral dx
\newcommand{\dx}{\mathrm{d}x}

% Alias for the Solution section header
\newcommand{\solution}{\textbf{Solution.}}

% Probability commands: Expectation, Variance, Covariance, Bias
%\newcommand{\E}{\mathbb{E}}
\newcommand{\Var}{\mathrm{Var}}
\newcommand{\Cov}{\mathrm{Cov}}
\newcommand{\Bias}{\mathrm{Bias}}
\newcommand\indep{\protect\mathpalette{\protect\independenT}{\perp}}
\def\independenT#1#2{\mathrel{\rlap{$#1#2$}\mkern2mu{#1#2}}}
\DeclareMathOperator{\Tr}{Tr}

\newcommand{\nPr}[2]{\,_{#1}P_{#2}}
\newcommand{\nCr}[2]{\,_{#1}C_{#2}}
%\binom{n}{r}

% Tilde
\newcommand{\textapprox}{\raisebox{0.5ex}{\texttildelow}}

\begin{document}

\maketitle

\pagebreak


% Problem 1
\begin{exercise}[\textbf{Probability/Mathematical Statistics}]

\begin{enumerate}[(a)]

% 1a
\item

\[
n^{-1} \log W_n = \frac{\log \left[(qr + (1-q)V_n) W_{n-1} \right]}{n}  = \frac{\log \left[qr + (1-q)V_n  \right]}{n} + \frac{ \log W_{n-1}}{n}
\]

%= \frac{\log \left[qrW_{n-1} + (1-q)V_n W_{n-1} \right]}{n}

Note that for \(q\) and \(r\) fixed, \(\log [qr + (1-q)V_n ] \) are i.i.d. random variables with mean \(\E \left( \log [qr + (1-q)V_1 ]  \right)\). Since \( \E \left| \log [qr + (1-q)V_1 ]  \right| < \infty\), by the Strong Law of Large Numbers

\[
\frac{\log \left[qr + (1-q)V_n  \right]}{n} \xrightarrow{a.s.} \E \log[qr + (1-q)V_1] = w(q).
\]

So the result follows if \(n^{-1}  \log W_{n-1} \xrightarrow{a.s.} 0 \). Note that

\[
 \frac{ \log W_{n-1}}{n} =  \frac{ \log  \left[(qr + (1-q)V_{n-1}) W_{n-2} \right] }{n} 
\]

% 1b
\item

\[
w(q) =  \E \log[qr + (1-q)V_1] =  \E \log[q(r - V_1) + V_1] 
\]

Let \(t \in (0,1)\). Let \(q_1, q_2 \in (0,1]\), and suppose without loss of generality \(q_1 \leq q_2\). We wish to show that

\begin{equation}\label{cvx.541a.hw6.5a}
w(tq_1+(1-t)q_2)\geq t w(q_1)+(1-t) w(q_2)
\end{equation}

\[
 \iff      \E \log[(tq_1+(1-t)q_2) r + [1-(tq_1+(1-t)q_2) ]V_1]  \geq t   \E \log[q_1r + (1-q_1)V_1]  +(1-t)  \E \log[q_2r + (1-q_2)V_1] 
 \]
 
 \[
 \iff      \E \log[tq_1r+(1-t)q_2 r + V_1  - tq_1V_1 -(1-t)q_2) V_1]  \geq t   \E \log[q_1r + (1-q_1)V_1]  +(1-t)  \E \log[q_2r + (1-q_2)V_1] 
 \]




% 1c
\item

\end{enumerate}

\end{exercise}

% Problem 2
\begin{exercise}[\textbf{Mathematical statistics, Bayesian}]

\begin{enumerate}[(a)]

% 2a
\item

% 2b
\item

\end{enumerate}

\end{exercise}

% Problem 3
\begin{exercise}[\textbf{Convergence; Wen says we don't need to worry about. From Trambak's exam; he took Analysis of Algorithms}]

\begin{enumerate}[(a)]

% 3a
\item

% 3b
\item

\end{enumerate}

\end{exercise}

% Problem 4
\begin{exercise}[\textbf{Convex Optimization}]

\begin{enumerate}[(a)]

% 4a
\item

% 4b
\item

\end{enumerate}

\end{exercise}

% Problem 5
\begin{exercise}[\textbf{High-Dimensional Statistics}] \textbf{Double-check solutions, comments from Jinchi grading homework}

\begin{enumerate}[(a)]

% 1a
\item Let 

\[
\boldsymbol{X} = \begin{bmatrix} \boldsymbol{x}_1^T \\ \vdots \\  \boldsymbol{x}_n^T   \end{bmatrix} \in \mathbb{R}^{n \times p}
\]

be the design matrix. Given that the mean is known to be 0, the covariance matrix is defined as 

\[
\boldsymbol{\Sigma} = \E \big(  \boldsymbol{X}^T \boldsymbol{X} \big)
\]

A natural unbiased estimator for \(\boldsymbol{\Sigma}\) is the sample covariance

\[
\widehat{\boldsymbol{\Sigma}} = \frac{1}{n} \big(\boldsymbol{X}^T \boldsymbol{X} \big).
\]

% 1b
\item We have different issues in each of these regimes.

\begin{enumerate}[(1)]

\item \textbf{\(p \leq n\) and \(p\) is roughly of the same order as \(n\):} there can be significant sampling error in estimating \(\widehat{\boldsymbol{\Sigma}}\) in this regime. \citet{Fan2008} showed that under Frobenius norm this estimator has a very slow convergence rate even if \(p < n\). Further, the expected value of its inverse is

\[
\E  \big(\widehat{\boldsymbol{\Sigma}}^{-1}\big) = \frac{n}{n-p-2}  \boldsymbol{\Sigma}^{-1}
\] 

\citep{Bai2011}, so this bias can be quite large if \(p \approx n\), even if \(p < n\). (A better method for estimating \( \boldsymbol{\Sigma}^{-1}\) directly is presented by \citet{Fan2008}.)

\item \textbf{\(p > n\) or even \(p \gg n\):} in that case \(\widehat{\boldsymbol{\Sigma}} = \frac{1}{n} \big(\boldsymbol{X}^T \boldsymbol{X} \big)\) will be rank-deficient and singular, even though the true covariance matrix will be nonsingular (and positive definite), so clearly \(\widehat{\boldsymbol{\Sigma}}\) will not be an ideal estimate. 

\end{enumerate}


% 1c
\item 

% Week 1 slides

\citet{Geman1980} showed that in the case of \(\boldsymbol{\Sigma} = I_p\),

\[
\lambda_{\text{max}} \big( \widehat{\boldsymbol{\Sigma}}  \big) \xrightarrow{a.s.} (1 + \gamma^{-1/2})^2 \text{ as } n/p \to \gamma \geq 1.
\]

Further, numerical studies that that \(\lambda_{\text{max}} \big( \widehat{\boldsymbol{\Sigma}}  \big) \) for \(n = 100\) typically ranges between 1.2 - 1.5 for \(p=5\), between 2.6 and 3 for \(p=50\), and between 10 and 10.5 for \(p = 500\). Of course, the correct maximum eigenvalue is 1 (since all eigenvalues of \( I_p\) are 1), so we see that covariance matrix estimation gets increasing unstable and inaccurate as \(p \gg n\). 

\

Regarding the limiting distribution of the largest eigenvalue \(\lambda_{\text{max}} \big( \widehat{\boldsymbol{\Sigma}}  \big)\), \citet{Johnstone2001} showed that

\[
\frac{n\lambda_{\text{max}} \big( \widehat{\boldsymbol{\Sigma}}  \big)  - \mu_{np} }{\sigma_{np}} \xrightarrow{D} \text{ Tracy-Widom law of order 1 as } n/p \to \gamma \geq 1
\]

where 

\[
\mu_{np} = ( \sqrt{n-1} + \sqrt{p})^2, \qquad \sigma_{np} = ( \sqrt{n-1} + \sqrt{p})(1/\sqrt{n-1} + 1/\sqrt{p})^{1/3}.
\]

\end{enumerate} 

\end{exercise}


\bibliographystyle{abbrvnat}
\bibliography{mybib2fin}
\end{document}