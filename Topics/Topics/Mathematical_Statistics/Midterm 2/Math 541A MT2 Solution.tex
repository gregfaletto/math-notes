\documentclass{article}

\usepackage{fancyhdr}
\usepackage{extramarks}
\usepackage{amsmath}
\usepackage{amsthm}
\usepackage{amsfonts}
\usepackage{tikz}
\usepackage{enumerate}
\usepackage{graphicx}
\graphicspath{ {images/} }
\usepackage[plain]{algorithm}
\usepackage{algpseudocode}
\usepackage[document]{ragged2e}
\usepackage{textcomp}
% \usepackage{amssymb}

\usetikzlibrary{automata,positioning}

%
% Basic Document Settings
%

\topmargin=-0.45in
\evensidemargin=0in
\oddsidemargin=0in
\textwidth=6.5in
\textheight=9.0in
\headsep=0.25in

\linespread{1.1}

\pagestyle{fancy}
\lhead{\hmwkAuthorName}
\chead{\hmwkClass\ (\hmwkClassInstructor): \hmwkTitle}
%\rhead{\firstxmark}
%\lfoot{\lastxmark}
\cfoot{\thepage}

\renewcommand\headrulewidth{0.4pt}
\renewcommand\footrulewidth{0.4pt}

\setlength\parindent{0pt}

%
% Create Problem Sections
%

%\newcommand{\enterProblemHeader}[1]{
%    \nobreak\extramarks{}{Question \arabic{#1} continued on next page\ldots}\nobreak{}
%    \nobreak\extramarks{Question \arabic{#1} (cont.)}{Question \arabic{#1} continued on next page\ldots}\nobreak{}
%}
%
%\newcommand{\exitProblemHeader}[1]{
%    \nobreak\extramarks{Question \arabic{#1} (cont.)}{Question \arabic{#1} continued on next page\ldots}\nobreak{}
%    \stepcounter{#1}
%    \nobreak\extramarks{Question \arabic{#1}}{}\nobreak{}
%}
%
%\setcounter{secnumdepth}{0}
%\newcounter{partCounter}
%\newcounter{homeworkProblemCounter}
%\setcounter{homeworkProblemCounter}{1}
%\nobreak\extramarks{Problem \arabic{homeworkProblemCounter}}{}\nobreak{}

%%
%% Homework Problem Environment
%%
%% This environment takes an optional argument. When given, it will adjust the
%% problem counter. This is useful for when the problems given for your
%% assignment aren't sequential. See the last 3 problems of this template for an
%% example.
%%
%\newenvironment{homeworkProblem}[1][-1]{
%    \ifnum#1>0
%        \setcounter{homeworkProblemCounter}{#1}
%    \fi
%    \section{Problem \arabic{homeworkProblemCounter}}
%    \setcounter{partCounter}{1}
%    \enterProblemHeader{homeworkProblemCounter}
%}{
%    \exitProblemHeader{homeworkProblemCounter}
%}

\theoremstyle{definition}
\newtheorem{exercise}{Exercise}
\newtheorem{definition}{Definition}
\newtheorem*{remark}{Remark}
\theoremstyle{definition}
\newtheorem{theorem}{Theorem}
\theoremstyle{definition}
\newtheorem{lemma}[theorem]{Lemma}
\theoremstyle{definition}
\newtheorem{corollary}{Corollary}[theorem]

%
% Homework Details
%   - Title
%   - Due date
%   - Class
%   - Section/Time
%   - Instructor
%   - Author
%

\newcommand{\hmwkTitle}{Homework 7}
\newcommand{\hmwkDueDate}{Apr. 26, 2019}
\newcommand{\hmwkClass}{Math 541A}
%\newcommand{\hmwkClassTime}{Section A}
\newcommand{\hmwkClassInstructor}{S. Heilman}
\newcommand{\hmwkAuthorName}{\textbf{G. Faletto} }

%\renewcommand{\subset}{\subseteq}
\renewcommand{\supset}{\supseteq}
\renewcommand{\epsilon}{\varepsilon}

\newcommand{\abs}[1]{\left|#1\right|}                   % Absolute value notation
\newcommand{\absf}[1]{|#1|}                             % small absolute value signs
\newcommand{\vnorm}[1]{\left|\left|#1\right|\right|}    % norm notation
\newcommand{\vnormf}[1]{||#1||}                         % norm notation, forced to be small
\newcommand{\im}[1]{\mbox{im}#1}                        % Pieces of English for math mode
\newcommand{\tr}[1]{\mbox{tr}#1}
\newcommand{\Proj}[1]{\mbox{Proj}#1}
\newcommand{\Vol}[1]{\mbox{Vol}#1}
\newcommand{\Z}{\mathbb{Z}}                             % Blackboard notation
\newcommand{\N}{\mathbb{N}}
\newcommand{\E}{\mathbb{E}}
\renewcommand{\P}{\mathbb{P}}
\newcommand{\R}{\mathbb{R}}
\newcommand{\C}{\mathbb{C}}
\newcommand{\Q}{\mathbb{Q}}
\newcommand{\figoneawidth}{.5\textwidth}                % Image formatting parameters
\newcommand{\lbreak}{\\}                                % Linebreak
\newcommand{\italicize}[1]{\textit {#1}}                % formatting commands for bibliography
%\newcommand{\embolden}[1]{\textbf {#1}}
\newcommand{\embolden}[1]{{#1}}
\newcommand{\undline}[1]{\underline {#1}}
\newcommand{\e}{\varepsilon}
\renewcommand{\epsilon}{\varepsilon}
%\renewcommand{:=}{=}

%
% Title Page
%

\title{
    \vspace{2in}
    \textmd{\textbf{Math 541A - Mathematical Statistics:\ \hmwkTitle}}\\
    \normalsize\vspace{0.1in}\small{Due\ on\ \hmwkDueDate}\\
    \vspace{0.1in}\large{\textit{Instructor: Dr. Steven Heilman\ }}
    \vspace{3in}
}

\author{Gregory Faletto}
\date{}

\renewcommand{\part}[1]{\textbf{\large Part \Alph{partCounter}}\stepcounter{partCounter}\\}

%
% Various Helper Commands
%

% Useful for algorithms
\newcommand{\alg}[1]{\textsc{\bfseries \footnotesize #1}}

% For derivatives
\newcommand{\deriv}[2]{\frac{\mathrm{d} #1}{\mathrm{d} #2}}

% For partial derivatives
\newcommand{\pderiv}[2]{\frac{\partial #1}{\partial #2}}

% Integral dx
\newcommand{\dx}{\mathrm{d}x}

% Alias for the Solution section header
\newcommand{\solution}{\textbf{Solution.}}

% Probability commands: Expectation, Variance, Covariance, Bias
%\newcommand{\E}{\mathbb{E}}
\newcommand{\Var}{\mathrm{Var}}
\newcommand{\Cov}{\mathrm{Cov}}
\newcommand{\Bias}{\mathrm{Bias}}
\newcommand\indep{\protect\mathpalette{\protect\independenT}{\perp}}
\def\independenT#1#2{\mathrel{\rlap{$#1#2$}\mkern2mu{#1#2}}}
\DeclareMathOperator{\Tr}{Tr}

\newcommand{\nPr}[2]{\,_{#1}P_{#2}}
\newcommand{\nCr}[2]{\,_{#1}C_{#2}}
%\binom{n}{r}

% Tilde
\newcommand{\textapprox}{\raisebox{0.5ex}{\texttildelow}}

\begin{document}

\maketitle

\pagebreak


% Problem 1
\begin{exercise}

\end{exercise}

% Problem 2
\begin{exercise}

\end{exercise}


% Problem 3
\begin{exercise}

First,

\[
\E( \overline{X}^2) = \E \left( \frac{1}{n} \sum_{i=1}^n X_i \right)^2 =  \frac{1}{n^2}\E \left(  \sum_{i=1}^n X_i^2 + 2 \sum_{1 \leq i < j \leq n} X_i X_j \right) =  \frac{1}{n^2}\left[\sum_{i=1}^n \E \left(  X_i^2 \right) + 2 \binom{n}{2} \mu^2 \right]
\]

\[
=  \frac{1}{n^2}\sum_{i=1}^n \left( \Var(X_i) + \E(X_i)^2 \right) +  \frac{2}{n^2} \frac{n(n-1)}{2}  \mu^2 =  \frac{1}{n^2}\cdot n  \left( \sigma^2 + \mu^2 \right) +  \frac{n-1}{n} \mu^2
\]

\[
=  \mu^2 + \frac{\sigma^2}{n}.
\]

Therefore since \(S^2\) is unbiased for \(\sigma^2\), \(Y := \overline{X}^2 - n^{-1} S^2\) is unbiased for \(\mu^2\):

\[
\E(Y) = \E \left(  \overline{X}^2 - n^{-1} S^2 \right) =  \mu^2 + \frac{\sigma^2}{n} - \frac{\sigma^2}{n} = \mu^2.
\]

Note that \(Y\) is a function of the complete sufficient statistic \(Z:= (\overline{X}, S^2)\); that is, \(Y = h(Z)\) where \(h\big( (t_1, t_2) \big) = t_1^2 - n^{-1} t_2\). Therefore by Lehmann-Scheffe, \(\E_\theta (Y \mid Z) = \E_\theta(h(Z) \mid Z) = h(Z) = Y\) is UMVU for \(\mu^2\).

\end{exercise}


% Problem 4
\begin{exercise}

\

\begin{center}
\noindent\fbox{
\parbox{0.9\textwidth}{

\begin{theorem}[\textbf{Rao-Blackwell; Theorem 6.4 in Math 541A notes}] Let \(Z\) be a sufficient statistic for \(\{f_\theta:\theta \in \Theta\}\) and let \(Y\) be an estimator for \(g(\theta)\). Define \(W:= \E_\theta(Y \mid Z)\). Let \(\theta \in \Theta\). Then



\[
\Var_\theta(W) \leq \Var_\theta (Y).
\]

Further, let \(r(\theta, y) < \infty\) and such that \(\ell(\theta, y)\) is convex in \(y\). Then

\[
r(\theta, W) \leq r(\theta, Y).
\]

\end{theorem}
}
}
\end{center}

\

\begin{proof}[Proof (just of risk part)] Note that since \(Z\) is sufficient, \(W\) does not depend on \(\theta\). By the Conditional Jensen's Inequality and using the convexity of \(\ell(\theta, y)\) in \(y\),

\[
\ell(\theta, w) = \ell(\theta, \E_{\tilde{\theta}}(Y \mid Z)) \leq \E_{\tilde{\theta}} [ \ell(\theta, Y) \mid Z].
\]

Take expectations of both sides to get

\[
\E_{\tilde{\theta}} \ell(\theta, w) =  r(\theta, W)  \leq \E_{\tilde{\theta}} \E_{\tilde{\theta}} [ \ell(\theta, Y) \mid Z] = \E_{\tilde{\theta}} \ell(\theta, Y) = r(\theta, Y).
\]

If \(\ell(\theta, y)\) is strictly convex in \(y\) then this inequality is strict, unless \(Y\) is a function of \(Z\). If \(Y\) is a function of \(Z\), then \(\E_\theta(Y \mid Z) = Y\), so \(W = Y\).

\end{proof}

\begin{proof}[Proof (Variance part, then risk part as consequence)]

First we will show that \(\mathrm{Var}_{\theta}(Y) \geq \mathrm{Var}_\theta( Y \mid Z)\), where \(Z\) is a sufficient statistic for \(g(\theta)\). Then we will show that this implies that \(r(\theta, Y \mid Z) \leq r(\theta, Y)\) when the loss function is mean squared error.
 
 \
 
 Let \(W := \E(Y \mid Z)\). Using the given identities and the Law of Total Variance,

\[
\mathrm{Var}_{\theta}(Y) = \E_\theta[\mathrm{Var}_\theta(Y|Z)]+\mathrm{Var}_\theta[\E_\theta(Y|Z)]= \E_\theta[\mathrm{Var}_\theta(Y|Z)]+\mathrm{Var}_\theta(W)
\]


Note that \(\mathrm{Var}_\theta(Y|Z)] \geq 0 \implies \E_\theta[\mathrm{Var}_\theta(Y|Z)] \geq 0\). Therefore we have


\begin{equation}\label{mathstats.541a.hw6.1a}
\mathrm{Var}_{\theta}(Y) \geq \mathrm{Var}_\theta(W)
\end{equation}




as desired. Next, let \(\mu = \E(Y)\). Then we have

\[
\E(Y - g(\theta))^2 = \E(Y - \mu + \mu - g(\theta))^2 = \E\big[(Y- \mu)^2 + (\mu - g(\theta))^2 + 2 (Y - \mu)(\mu - g(\theta)) \big]
\]

\[
 = \E\big[(Y- \mu)^2 \big] + \E \big[ (\mu - g(\theta))^2 \big] + 2 \E \big[ (Y - \mu)(\mu - g(\theta)) \big]  = \Var(Y) + (\mu - g(\theta))^2 
\]

Since \( \E(Y - g(\theta))^2 = r(\theta, Y) \), we have

\begin{equation}\label{mathstats.541a.hw6.1b}
 \Var(Y) = r(\theta, Y)   - (\mu - g(\theta))^2 
\end{equation}

where \( \E \big[ (\mu - g(\theta))^2 \big] = (\mu - g(\theta))^2 \) because both quantities are constants. Similarly, since \(\E(W) = \E[\E_\theta (Y \mid Z)]  = \E(Y) = \mu\), we have

\[
\E(W - g(\theta))^2 = \E(W - \mu + \mu - g(\theta))^2 = \E\big[(W- \mu)^2 + (\mu - g(\theta))^2 + 2 (W - \mu)(\mu - g(\theta)) \big]
\]

\[
 = \E\big[(W- \mu)^2 \big] + \E \big[ (\mu - g(\theta))^2 \big] + 2 \E \big[ (W - \mu)(\mu - g(\theta)) \big]  = \Var(W) + (\mu - g(\theta))^2 
\]

\begin{equation}\label{mathstats.541a.hw6.1c}
\iff  \Var(W) = r(\theta, W)  - (\mu - g(\theta))^2 .
\end{equation}

Finally, substituting (\ref{mathstats.541a.hw6.1b}) and (\ref{mathstats.541a.hw6.1c}) into (\ref{mathstats.541a.hw6.1a}) yields

\[
r(\theta, Y)   - (\mu - g(\theta))^2 \geq r(\theta, W)  - (\mu - g(\theta))^2 \iff r(\theta, Y)  \geq  r(\theta, W).
\]


\end{proof}


\end{exercise}



\end{document}