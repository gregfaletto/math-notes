\documentclass{article}

\usepackage{fancyhdr}
\usepackage{extramarks}
\usepackage{amsmath}
\usepackage{amsthm}
\usepackage{amsfonts}
\usepackage{tikz}
\usepackage[plain]{algorithm}
\usepackage{algpseudocode}

\usetikzlibrary{automata,positioning}

%
% Basic Document Settings
%

\topmargin=-0.45in
\evensidemargin=0in
\oddsidemargin=0in
\textwidth=6.5in
\textheight=9.0in
\headsep=0.25in

\linespread{1.1}

\pagestyle{fancy}
\lhead{\hmwkAuthorName}
\chead{\hmwkClass\ : \hmwkTitle} %(\hmwkClassInstructor\) \hmwkClassTime): 
\rhead{\firstxmark}
\lfoot{\lastxmark}
\cfoot{\thepage}

\renewcommand\headrulewidth{0.4pt}
\renewcommand\footrulewidth{0.4pt}

\setlength\parindent{0pt}

%
% Create Problem Sections
%

\newcommand{\enterProblemHeader}[1]{
    \nobreak\extramarks{}{Problem (\alph{#1}) continued on next page\ldots}\nobreak{}
    \nobreak\extramarks{Problem (\alph{#1}) (continued)}{Problem (\alph{#1}) continued on next page\ldots}\nobreak{}
}

\newcommand{\exitProblemHeader}[1]{
    \nobreak\extramarks{Problem (\alph{#1}) (continued)}{Problem (\alph{#1}) continued on next page\ldots}\nobreak{}
    \stepcounter{#1}
    \nobreak\extramarks{Problem (\alph{#1})}{}\nobreak{}
}

\setcounter{secnumdepth}{0}
\newcounter{partCounter}
\newcounter{homeworkProblemCounter}
\setcounter{homeworkProblemCounter}{1}
\nobreak\extramarks{Problem (\alph{homeworkProblemCounter})}{}\nobreak{}

%
% Homework Problem Environment
%
% This environment takes an optional argument. When given, it will adjust the
% problem counter. This is useful for when the problems given for your
% assignment aren't sequential. See the last 3 problems of this template for an
% example.
%
\newenvironment{homeworkProblem}[1][-1]{
    \ifnum#1>0
        \setcounter{homeworkProblemCounter}{#1}
    \fi
    \section{Problem (\alph{homeworkProblemCounter})}
    \setcounter{partCounter}{1}
    \enterProblemHeader{homeworkProblemCounter}
}{
    \exitProblemHeader{homeworkProblemCounter}
}

%
% Homework Details
%   - Title
%   - Due date
%   - Class
%   - Section/Time
%   - Instructor
%   - Author
%

\newcommand{\hmwkTitle}{Test Exercise\ \#5}
%\newcommand{\hmwkDueDate}{February 12, 2014}
\newcommand{\hmwkClass}{MOOC Econometrics}
%\newcommand{\hmwkClassTime}{Section A}
%\newcommand{\hmwkClassInstructor}{Professor Isaac Newton}
\newcommand{\hmwkAuthorName}{\textbf{Greg Faletto}} % \and \textbf{Davis Josh}}

%
% Title Page
%

\title{
    \vspace{2in}
    \textmd{\textbf{\hmwkClass:\ \hmwkTitle}}\\
    %\normalsize\vspace{0.1in}\small{Due\ on\ \hmwkDueDate\ at 3:10pm}\\
    %\vspace{0.1in}\large{\textit{\hmwkClassInstructor\ \hmwkClassTime}}
    \vspace{3in}
}

\author{\hmwkAuthorName}
\date{}

\renewcommand{\part}[1]{\textbf{\large Part \Alph{partCounter}}\stepcounter{partCounter}\\}

%
% Various Helper Commands
%

% Useful for algorithms
\newcommand{\alg}[1]{\textsc{\bfseries \footnotesize #1}}

% For derivatives
\newcommand{\deriv}[1]{\frac{\mathrm{d}}{\mathrm{d}x} (#1)}

% For partial derivatives
\newcommand{\pderiv}[2]{\frac{\partial}{\partial #1} (#2)}

% Integral dx
\newcommand{\dx}{\mathrm{d}x}

% Alias for the Solution section header
\newcommand{\solution}{\textbf{\large Solution}}

% Probability commands: Expectation, Variance, Covariance, Bias
\newcommand{\E}{\mathrm{E}}
\newcommand{\Var}{\mathrm{Var}}
\newcommand{\Cov}{\mathrm{Cov}}
\newcommand{\Bias}{\mathrm{Bias}}

\begin{document}

\maketitle

\pagebreak

\begin{homeworkProblem}
    Consider again the application in lecture 5.5, where we have analyzed response to a direct mailing using the following logit specification: 

\[
\Pr[resp_i = 1] = \frac{\exp(\beta_0 + \beta_1male_i + \beta_2active_i + \beta_3age_i + \beta_4(age_i/10)^2)}{1 + \exp(\beta_0 + \beta_1male_i + \beta_2active_i + \beta_3age_i + \beta_4(age_i/10)^2)}
\]

for \( i = 1, \ldots , 925\). The maximum likelihood estimates of the parameters are given by


\begin{center}
\begin{tabular}{l l l l l}
	\hline
	Variable & Coefficient & Std. Error & t-value & p-value \\
	\hline
	\(Intercept\) & -2.488 & 0.890 & -2.796 & 0.005 \\
	\(Male\) & 0.954 & 0.158 & 6.029 & 0.000 \\
	\(Active\) & 0.914 & 0.185 & 4.945 & 0.000 \\
	\(Age\) & 0.070 & 0.036 & 1.964 & 0.050 \\
	\((Age/10)^2\) & -0.069 & 0.034 & -2.015 & 0.044 \\
	\hline
\end{tabular}
\end{center}


The marginal effect of activity status is defined as

\[
\frac{\partial \Pr[resp_i = 1]}{\partial active_i} = \Pr[resp_i = 1]\Pr[resp_i =0]\beta_2.
\]


We could use this result to construct an activity status elasticity

\[
\frac{\partial \Pr[resp_i = 1]}{\partial active_i}\frac{active_i}{\Pr[resp_i = 1]} = \Pr[resp_i = 0]active_i\beta_2.
\]


Use this result to compute the elasticity effect of active status for a 50-year-old active male customer. Do the same for a 50-year-old inactive male customer.
\\

    \solution
    \\
   
    \[
    \Pr[resp_i = 0] = 1 - \Pr[resp_i = 1] 
    \]
    \[
= 1 - \frac{\exp(\beta_0 + \beta_1male_i + \beta_2active_i + \beta_3age_i + \beta_4(age_i/10)^2)}{1 + \exp(\beta_0 + \beta_1male_i + \beta_2active_i + \beta_3age_i + \beta_4(age_i/10)^2)}
    \]
    
        \[
    =  \frac{1 + \exp(\beta_0 + \beta_1male_i + \beta_2active_i + \beta_3age_i + \beta_4(age_i/10)^2) - \exp(\beta_0 + \beta_1male_i + \beta_2active_i + \beta_3age_i + \beta_4(age_i/10)^2)}{1 + \exp(\beta_0 + \beta_1male_i + \beta_2active_i + \beta_3age_i + \beta_4(age_i/10)^2)}
    \]
    \[
    =  \frac{1}{1 + \exp(\beta_0 + \beta_1male_i + \beta_2active_i + \beta_3age_i + \beta_4(age_i/10)^2)}
    \]
    For a 50-year-old active male customer, \(male_i = 1, active_i = 1, age_i = 50, (age_i/10)^2 = (50/10)^2 = 25.\)
    \\
    
    Therefore

\[
elasticity = \Pr[resp_i = 0]active_i\beta_2
\]
\[
= \frac{1}{1 + \exp(\beta_0 + 1\beta_1 + 1\beta_2 + 50\beta_3 + 25\beta_4}\cdot 1 \cdot \beta_2
\]

\pagebreak

Plugging in the given coefficients, we have

\[
elasticity = \frac{1}{1 + \exp(-2.488 + 1\cdot 0.954 + 1\cdot 0.914 + 50\cdot 0.070 + 25\cdot -0.069}\cdot 1 \cdot 0.914
\]

\noindent\fbox{
    \parbox{\textwidth}{
\[
\approx \mathbf{0.2189734}
\]
}
}
\\


For a 50-year-old inactive male customer, \(male_i = 1, active_i = 0, age_i = 50, (age_i/10)^2 = (50/10)^2 = 25.\)


Therefore

\[
elasticity = \Pr[resp_i = 0]active_i\beta_2
\]

\[
= \frac{1}{1 + \exp(\beta_0 + 1\beta_1 + 1\beta_2 + 50\beta_3 + 25\beta_4}\cdot 0 \cdot \beta_2
\]

\noindent\fbox{
    \parbox{\textwidth}{
\[
= \mathbf{0}
\]
}
}
    
\end{homeworkProblem}

%\pagebreak

\begin{homeworkProblem}
    The activity status variable is only a dummy variable and hence it can take only two values. It is therefore better to define the elasticity as
    
    \[
   \frac{ \Pr[resp_i = 1 | active_i = 1] - \Pr[resp_i = 1 | active_i = 0]}{\Pr[resp_i = 1 | active_i = 0]}
   \]
   
   Show that you can simplify the expression for the elasticity as
   
   \[
   (\exp(\beta_2) - 1)\Pr[resp_i = 0 | active_i =1].
   \]
   
   

    \solution
    \\
    
    \[
    \Pr[resp_i = 1 | active_i = 1] = \frac{\exp(\beta_0 + \beta_1 male_i + \beta_2 \cdot 1 + \beta_3 age_i + \beta_4 (age_i/10)^2)}{1 + \exp(\beta_0 + \beta_1 male_i + \beta_2 \cdot 1 + \beta_3 age_i + \beta_4 (age_i/10)^2)}
    \]
    
        \[
    \Pr[resp_i = 1 | active_i = 0] = \frac{\exp(\beta_0 + \beta_1 male_i +  \beta_3 age_i + \beta_4 (age_i/10)^2)}{1 + \exp(\beta_0 + \beta_1 male_i + \beta_3 age_i + \beta_4 (age_i/10)^2)}
    \]
    
        \[
    \Pr[resp_i = 0 | active_i = 1] = 1 - \Pr[resp_i = 1 | active_i = 1] 
    \]
    
    \[
    = 1- \frac{\exp(\beta_0 + \beta_1 male_i + \beta_2 \cdot 1 + \beta_3 age_i + \beta_4 (age_i/10)^2)}{1 + \exp(\beta_0 + \beta_1 male_i + \beta_2 \cdot 1 + \beta_3 age_i + \beta_4 (age_i/10)^2)}
    \]
    
            \[
    = \frac{1 + \exp(\beta_0 + \beta_1 male_i + \beta_2 \cdot 1 + \beta_3 age_i + \beta_4 (age_i/10)^2) - \exp(\beta_0 + \beta_1 male_i + \beta_2 \cdot 1 + \beta_3 age_i + \beta_4 (age_i/10)^2)}{1 + \exp(\beta_0 + \beta_1 male_i + \beta_2 + \beta_3 age_i + \beta_4 (age_i/10)^2)}
    \]
    
        \[
    = \frac{1}{1 + \exp(\beta_0 + \beta_1 male_i + \beta_2 + \beta_3 age_i + \beta_4 (age_i/10)^2)}
    \]
   \\
   
   %\pagebreak
    
    Let \( \beta_{-2} = [\beta_0 \; \beta_1 \; \beta_3 \; \beta_4 ]' \). Let \( X_{-2} = [1 \; male_i \; age_i \; (age_i/10)^2 ]'. \) Then
    
    \[
    {\beta_{-2}}' X_{-2} = \beta_0 + \beta_1 male_i + \beta_3 age_i + \beta_4 (age_i/10)^2
    \]
    
    and
    
        \[
    {\beta_{-2}}' X_{-2} + \beta_2 = \beta_0 + \beta_1 male_i + \beta_2 + \beta_3 age_i + \beta_4 (age_i/10)^2
    \]
    
    so
    
    \[
    \Pr[resp_i = 1 | active_i = 1] = \frac{\exp({\beta_{-2}}'X_{-2} + \beta_2)}{(1+\exp({\beta_{-2}}'X_{-2} + \beta_2)} = \frac{\exp({\beta_{-2}}'X_{-2})\exp(\beta_2)}{1+\exp({\beta_{-2}}'X_{-2})\exp(\beta_2)}
    \]
    \\
    and
    
    \[
    \Pr[resp_i = 1 | active_i = 0] = \frac{\exp({\beta_{-2}}'X_{-2})}{1 + \exp({\beta_{-2}}'X_{-2})}
\]

Therefore

\[
elasticity = \frac{\frac{\exp({\beta_{-2}}'X_{-2})\exp(\beta_2)}{1+\exp({\beta_{-2}}'X_{-2})\exp(\beta_2)} - \frac{\exp({\beta_{-2}}'X_{-2})}{1 + \exp({\beta_{-2}}'X_{-2})}}{\frac{\exp({\beta_{-2}}'X_{-2})}{1 + \exp({\beta_{-2}}'X_{-2})}}
\]

\[
= \bigg(\frac{\exp({\beta_{-2}}'X_{-2})\exp(\beta_2)}{1+\exp({\beta_{-2}}'X_{-2})\exp(\beta_2)} - \frac{\exp({\beta_{-2}}'X_{-2})}{1 + \exp({\beta_{-2}}'X_{-2})}\bigg) \cdot \frac{1 + \exp({\beta_{-2}}'X_{-2})}{\exp({\beta_{-2}}'X_{-2})}
\]

\[
= \bigg(\frac{\exp(\beta_2)}{1+\exp({\beta_{-2}}'X_{-2})\exp(\beta_2)} - \frac{1}{1 + \exp({\beta_{-2}}'X_{-2})}\bigg) \cdot \big[1 + \exp({\beta_{-2}}'X_{-2}) \big]
\]

\[
= \frac{\exp(\beta_2)(1 + \exp({\beta_{-2}}'X_{-2}))}{1+\exp({\beta_{-2}}'X_{-2})\exp(\beta_2)} - 1
\]

\[
= \frac{\exp(\beta_2) + \exp(\beta_2) \exp({\beta_{-2}}'X_{-2})) - 1 - \exp({\beta_{-2}}'X_{-2})\exp(\beta_2) }{1+\exp({\beta_{-2}}'X_{-2})\exp(\beta_2)}
\]

\[
= \frac{\exp(\beta_2) - 1 }{1+\exp({\beta_{-2}}'X_{-2})\exp(\beta_2)}
\]

\[
= \big( \exp(\beta_2) - 1 \big)  \cdot \frac{1}{1+\exp({\beta_{-2}}'X_{-2} + \beta_2)}
\]

\[
= \big( \exp(\beta_2) - 1 \big)  \cdot \frac{1}{1+\exp(\beta_0 + \beta_1 male_i + \beta_2 + \beta_3 age_i + \beta_4 (age_i/10)^2)}
\]

\noindent\fbox{
    \parbox{\textwidth}{
\[
= \mathbf { \big( \exp(\beta_2) - 1 \big)  \cdot  \Pr[resp_i = 0 | active_i = 1]}
\]
}
}
    
\end{homeworkProblem}

\pagebreak

\begin{homeworkProblem}

Use the formula in (b) to compute the activity elasticity of a 50-year-old male active customer.
\\
\\
    \solution
    
   \[
elasticity = \big( \exp(\beta_2) - 1 \big)  \cdot \frac{1}{1+\exp(\beta_0 + \beta_1 male_i + \beta_2 + \beta_3 age_i + \beta_4 (age_i/10)^2)}
\] 

    For a 50-year-old male active customer, \(male_i = 1, active_i = 1, age_i = 50, (age_i/10)^2 = (50/10)^2 = 25.\)
    \\

Plugging in our variable values and the given coefficients, we have

   \[
elasticity = \big( \exp(0.914) - 1 \big)  \cdot \frac{1}{1+\exp(-2.488 + 0.954 \cdot 1 + 0.914 + 0.070 \cdot 50+ -0.069 \cdot 25)}
\] 

\noindent\fbox{
    \parbox{\textwidth}{
\[
\approx \mathbf{0.3579951}
\]
}
}

\end{homeworkProblem}

\end{document}
