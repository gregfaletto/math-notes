\documentclass{article}

\usepackage{fancyhdr}
\usepackage{extramarks}
\usepackage{amsmath}
\usepackage{amsthm}
\usepackage{amsfonts}
\usepackage{tikz}
\usepackage[plain]{algorithm}
\usepackage{algpseudocode}

\usepackage{verbatim}
\usepackage{graphicx}
\graphicspath{ {images/} }

\usetikzlibrary{automata,positioning}

%
% Basic Document Settings
%

\topmargin=-0.45in
\evensidemargin=0in
\oddsidemargin=0in
\textwidth=6.5in
\textheight=9.0in
\headsep=0.25in

\linespread{1.1}

\pagestyle{fancy}
\lhead{\hmwkAuthorName}
\chead{\hmwkClass\ : \hmwkTitle} %(\hmwkClassInstructor\) \hmwkClassTime): 
\rhead{\firstxmark}
\lfoot{\lastxmark}
\cfoot{\thepage}

\renewcommand\headrulewidth{0.4pt}
\renewcommand\footrulewidth{0.4pt}

\setlength\parindent{0pt}

%
% Create Problem Sections
%

\newcommand{\enterProblemHeader}[1]{
    \nobreak\extramarks{}{Problem (\alph{#1}) continued on next page\ldots}\nobreak{}
    \nobreak\extramarks{Problem (\alph{#1}) (continued)}{Problem (\alph{#1}) continued on next page\ldots}\nobreak{}
}

\newcommand{\exitProblemHeader}[1]{
    \nobreak\extramarks{Problem (\alph{#1}) (continued)}{Problem (\alph{#1}) continued on next page\ldots}\nobreak{}
    \stepcounter{#1}
    \nobreak\extramarks{Problem (\alph{#1})}{}\nobreak{}
}

\setcounter{secnumdepth}{0}
\newcounter{partCounter}
\newcounter{homeworkProblemCounter}
\setcounter{homeworkProblemCounter}{1}
\nobreak\extramarks{Problem (\alph{homeworkProblemCounter})}{}\nobreak{}

%
% Homework Problem Environment
%
% This environment takes an optional argument. When given, it will adjust the
% problem counter. This is useful for when the problems given for your
% assignment aren't sequential. See the last 3 problems of this template for an
% example.
%
\newenvironment{homeworkProblem}[1][-1]{
    \ifnum#1>0
        \setcounter{homeworkProblemCounter}{#1}
    \fi
    \section{Problem (\alph{homeworkProblemCounter})}
    \setcounter{partCounter}{1}
    \enterProblemHeader{homeworkProblemCounter}
}{
    \exitProblemHeader{homeworkProblemCounter}
}

%
% Homework Details
%   - Title
%   - Due date
%   - Class
%   - Section/Time
%   - Instructor
%   - Author
%

\newcommand{\hmwkTitle}{Test Exercise\ \#4}
%\newcommand{\hmwkDueDate}{February 12, 2014}
\newcommand{\hmwkClass}{MOOC Econometrics}
%\newcommand{\hmwkClassTime}{Section A}
%\newcommand{\hmwkClassInstructor}{Professor Isaac Newton}
\newcommand{\hmwkAuthorName}{\textbf{Greg Faletto}} % \and \textbf{Davis Josh}}

%
% Title Page
%

\title{
    \vspace{2in}
    \textmd{\textbf{\hmwkClass:\ \hmwkTitle}}\\
    %\normalsize\vspace{0.1in}\small{Due\ on\ \hmwkDueDate\ at 3:10pm}\\
    %\vspace{0.1in}\large{\textit{\hmwkClassInstructor\ \hmwkClassTime}}
    \vspace{3in}
}

\author{\hmwkAuthorName}
\date{}

\renewcommand{\part}[1]{\textbf{\large Part \Alph{partCounter}}\stepcounter{partCounter}\\}

%
% Various Helper Commands
%

% Useful for algorithms
\newcommand{\alg}[1]{\textsc{\bfseries \footnotesize #1}}

% For derivatives
\newcommand{\deriv}[1]{\frac{\mathrm{d}}{\mathrm{d}x} (#1)}

% For partial derivatives
\newcommand{\pderiv}[2]{\frac{\partial}{\partial #1} (#2)}

% Integral dx
\newcommand{\dx}{\mathrm{d}x}

% Alias for the Solution section header
\newcommand{\solution}{\textbf{\large Solution}}

% Probability commands: Expectation, Variance, Covariance, Bias
\newcommand{\E}{\mathrm{E}}
\newcommand{\Var}{\mathrm{Var}}
\newcommand{\Cov}{\mathrm{Cov}}
\newcommand{\Bias}{\mathrm{Bias}}

\begin{document}

\maketitle

\pagebreak

\section{Questions}

To run a study on the effect of a new diet a researcher runs a survey. The three most important questions in this survey are: (1) What was your weight one year ago? (2) What is your current weight? (3) Did you follow this diet in the past year?
\\

We denote the anwers of individual \(i = 1, \ldots, n\) to the first two questions with \(y_{i0}\) and \(y_{i1}\), the answer to the third question is denoted by \(d_i\) (where \(d_i = 1\) if the diet was followed). Furthermore, some background characteristics of the respondents are collected. These characteristics are combined in the vector \(x_i\) . Assume that all respondents are perfectly able to correctly answer the questions in the survey and do so truthfully.

\begin{homeworkProblem}

First of all the researcher uses OLS to estimate the parameters of the model

\[
y_{i1} - y_{i0} = \alpha + \beta d_i + \gamma y_{i0} + {x_i}' \delta + \epsilon_i
\]

The OLS estimator for \(\beta\) is possibly not consistent as the variable \(d_i\) may be endogenous. Clearly explain why this may be the case. Indicate whether your reason would lead OLS to overestimate or underestimate the true effect of the diet.
\\

 \solution
\\
    
Individuals decide for themselves whether they will go on the diet. Whether or not an individual followed this diet might be correlated with how health-conscious the individual is in general, which would also be correlated with how much weight they lost. I.e., if an individual is more health-conscious in general, they may have followed this diet but also engaged in other activities that would help their weight loss (like exercise).
\\

My reason would lead OLS to \textit{overestimate} the true effect of the diet. The causal benefits of the other activities these individuals would be engaging in (e.g. exercise) would be "absorbed" into the coefficient for whether they followed the diet. Including a variable like this in an OLS model would likely decrease the absolute value of the OLS coefficient for whether they followed the diet.
    
\end{homeworkProblem}

\begin{homeworkProblem}
The researcher finds out that in some regions of the country the diet was promoted via door-to-door advertising. The researcher manages to construct a variable \(z_i\) that indicates whether individual \(i\) does (\(z_i = 1\)) or does not (\(z_i = 0\)) live in a region in which the diet was advertised.
\\

In general there are two important conditions for variables \(Z\) to be useful as instruments. In formal terms these conditions are \(\frac{1}{n}Z' \epsilon \rightarrow 0\) and \(\frac{1}{n}Z'X \rightarrow Q \neq 0\) as the sample size \(n\) grows large. Rephrase these two conditions
in words in the context of this application for the above mentioned advertising variable (no formulas!).
\\

\pagebreak

    \solution
    \\  
    
\(\frac{1}{n}Z' \epsilon \rightarrow 0\): this means that \(Z\) and \(\epsilon\) are not correlated. In the context of this application, it means that whether or not an individual lives in a region where the diet was promoted does not explain any of the variation in the weight loss data that is left unexplained by a 2SLS model using a constant, an instrument for \(d_i\) based on \(Z\), \(y_i\), and \(x_i\) (\(\epsilon\)). In other words, advertisers did not select the regions where the ads were placed based on the average height or weight of the people who live in the region. We can verify this assumption using a Sargan test.
\\

\(\frac{1}{n}Z'X \rightarrow Q \neq 0\): This means that \(X\) and \(Z\) are sufficiently correlated. In the context of this experiment, it means that whether or not an individual lives in a region where the diet was promoted is correlated at least somewhat with whether they followed the diet (the stronger the correlation, the better). Individuals who live in regions where the diet was advertised should follow the diet at a higher rate than individuals elsewhere.

\end{homeworkProblem}

%\pagebreak

\begin{homeworkProblem}

For both assumptions in (b), indicate whether it can be tested statistically given the available variables. If yes, indicate how. If no, why not?
\\

    \solution
    \\
    
We would test the first condition (\(\frac{1}{n}Z' \epsilon \rightarrow 0\)) by using a Sargan test. In this case that would mean the following procedure:

\begin{itemize}

\item{Setup}

\begin{itemize}

\item{Model: \(y = X\beta + \epsilon\)}

\item{Explanatory variables: \(X = d_i\) (potentially endogenous), a constant, \(y_{i0}\), \(x_i\) (exogenous)}

\item{Instruments: \(Z = z_i\), a constant, \(y_{i0}\), \(x_i\)}

\item{Null hypothesis (\(H_0\)): Correlation of \(Z\) and \(\epsilon\) equals 0.}

\item{Test Procedure: Rewrite \(H_0\): \(\delta = 0\) in \(\epsilon = Z\delta + \xi\). \(\epsilon\) cannot be observed, so estimate \(\epsilon\) using 2SLS.}

\end{itemize}

\item{Use \(z_i\) (whether or no the individual lives in a region where the diet was advertised) to obtain the 2SLS estimator \(b_{2SLS}\) for \(\beta\).}

\begin{itemize}

\item{Regress \(d_i\) on \(Z\) (including \(z_i\), a constant, \(y_{i0}\), and \(x_i\)) to get what part of \(d_i\) is explained by \(Z\). (i.e., in the model \(d_i = Z\gamma + \eta\), so the part of \(d_i\) that is explained by \(Z\) is \(Z\gamma = \hat{d}\).}

\item{Obtain \(b_{2SLS}\) by regressing \(y\) on \(\hat{d}\).}

\end{itemize}

\item{Calculate \(e_{2SLS} = y - \hat{d}b_{2SLS}\).}

\item{Regress \(e_{2SLS}\) on \(Z\).}

\item{Use a significance test: \(nR^2 \approx \chi^2(m - k)\)}

\end{itemize}

Unfortunately, \textbf{this test cannot be completed for \(z_i\)}. One reason is because \textbf{we don't have enough instruments for a Sargan test}. We would need "too many" instruments in order to carry out the Sargan test (the number of instruments \(m\) would have to be greater than the number of explanatory variables \(k\) in order to calculate the value of a \(\chi^2\) statistic with \(m - k\) degrees of freedom).
\\

Another issue arises in the step of regressing \(d_i\) on \(Z\) to obtain \(b_{2SLS}\). \textbf{\(d_i\) is a categorical (dummy) variable, so it doesn't make sense to use a linear regression to predict \(d_i\)}. Specifically, this would violate assumption A1; clearly, the data generating process for \(d_i\) is not \(d_i = \alpha + \beta x_i + \epsilon_i \) since \(d_i\) is not continuous.
\\

We would test the second condition (\(\frac{1}{n}Z'X \rightarrow Q \neq 0\)) by checking the significance of instruments in the first stage regression of 2SLS (Step 1 above). Again, this regression is not possible to complete because it violates assumption A1, since \(d_i\) is a categorical variable. Therefore, \textbf{the second condition can also not be tested statistically given the available variables.}
  
\end{homeworkProblem}

%\pagebreak
  
\begin{homeworkProblem}
  
Suppose that \(z_i\) satisfies the conditions in (b) and suppose that \(z_i\) is uncorrelated with \(y_{i0}\) and \(x_i\). In this case the 2SLS-estimator for \(\beta\) in the model \(y_{i1} - y_{i0} = \alpha + \beta d_i + \eta_i\) is consistent when a constant and \(z_i\) are used as instruments.
\\

Show that we can write this 2SLS estimator for \(\beta\) in terms of simple sample averages. You can use the following averages:

\begin{enumerate}

\item{Average weight change over all individuals: \( \Delta \)}
\item{Average weight change over individuals with \(z_i = 1\): \( \Delta^1\)}
\item{Average weight change over individuals with \(z_i = 0\): \( \Delta^0\)}
\item{Proportion of people taking the diet: \(\bar{d}\)}
\item{Proportion of people with \(z_i = 1\) taking the diet: \(\bar{d}^1\)}
\item{Proportion of people with \(z_i = 0\) taking the diet: \(\bar{d}^0\)}

\end{enumerate}

To further explain the notation, for example:

\[
\bar{d}^1 = \frac{1}{\sum_{i=1}^{n}z_i} \sum_{i=1}^{n}z_i d_i
\]

\textbf{Hint:} start with the formula: \((Z'X)^{-1}Z'y.\)
\\
  \pagebreak
  
  \solution
  \\
  

  
\[
\Delta = \frac{1}{n}\sum_{i = 1}^{n}y_{i1} - y_{i0}
\]

\[
\Delta^1 = \frac{1}{\sum_{i=1}^n z_i} \sum_{i = 1}^{n}z_i(y_{i1} - y_{i0})
\]

\[
\Delta^0 = \frac{1}{\sum_{i=1}^n 1 - z_i} \sum_{i = 1}^{n}(1 - z_i)(y_{i1} - y_{i0})
\]

\[
\bar{d} = \frac{1}{n}\sum_{i=1}^{n}d_i
\]

\[
\bar{d}^1 = \frac{1}{\sum_{i=1}^{n}z_i} \sum_{i=1}^{n}z_i d_i
\]

\[
\bar{d}^0 = \frac{1}{\sum_{i=1}^{n}(1 - z_i)} \sum_{i=1}^{n}(1 - z_i) d_i
\]
\\

\begin{center}
\noindent\rule{4cm}{0.4pt}
\end{center}

\[
Z = {\begin{bmatrix}
1 & z_1\\
1 & z_2 \\
\vdots & \vdots \\
1 & z_n 
\end{bmatrix}}
\]

\[
X = {\begin{bmatrix}
1 & d_1\\
1 & d_2 \\
\vdots & \vdots \\
1 & d_n 
\end{bmatrix}}
\]

\[
y = {\begin{bmatrix}
y_{11} - y_{10} \\
y_{21} - y_{20} \\
\vdots \\
y_{n1} - y_{n0}
\end{bmatrix}}
\]

\[
\begin{bmatrix}
\hat{\alpha} \\
\hat{\beta}
\end{bmatrix} = (Z'X)^{-1}Z'y = 
{\begin{bmatrix}
n & \sum_{i=1}^{n}d_i \\
\sum_{i=1}^{n}z_i & \sum_{i=1}^{n}z_i d_i
\end{bmatrix}}^{-1} Z'y
\]

\[
= {\begin{bmatrix}
n & \sum_{i=1}^{n}d_i \\
\sum_{i=1}^{n}z_i & \sum_{i=1}^{n}z_i d_i
\end{bmatrix}}^{-1} 
\begin{bmatrix}
\sum_{i=1}^{n}y_{i1} - y_{i0} \\
\sum_{i=1}^{n}z_i(y_{i1} - y_{i0})
\end{bmatrix}
\]

\[
{\begin{bmatrix}
n & \sum_{i=1}^{n}d_i \\
\sum_{i=1}^{n}z_i & \sum_{i=1}^{n}z_i d_i
\end{bmatrix}}^{-1} = \frac{1}{n \sum_{i=1}^{n}z_i d_i - \sum_{i=1}^{n}d_i \sum_{i=1}^{n}z_i} 
\begin{bmatrix}
\sum_{i=1}^{n}z_i d_i & - \sum_{i=1}^{n}d_i \\
- \sum_{i=1}^{n}z_i & n
\end{bmatrix}
\]

  \pagebreak

\[
\implies {\begin{bmatrix}
n & \sum_{i=1}^{n}d_i \\
\sum_{i=1}^{n}z_i & \sum_{i=1}^{n}z_i d_i
\end{bmatrix}}^{-1} 
\begin{bmatrix}
\sum_{i=1}^{n}y_{i1} - y_{i0} \\
\sum_{i=1}^{n}z_i(y_{i1} - y_{i0})
\end{bmatrix} =
\]

\[ \frac{1}{n \sum_{i=1}^{n}z_i d_i - \sum_{i=1}^{n}d_i \sum_{i=1}^{n}z_i} 
\begin{bmatrix}
\sum_{i=1}^{n}z_i d_i & - \sum_{i=1}^{n}d_i \\
- \sum_{i=1}^{n}z_i & n
\end{bmatrix} \begin{bmatrix}
\sum_{i=1}^{n}y_{i1} - y_{i0} \\
\sum_{i=1}^{n}z_i(y_{i1} - y_{i0})
\end{bmatrix}
\]
\\

The first row of this matrix is \(\hat{\alpha}\) and the second row of this matrix is \(\hat{\beta}\). Hence we are only interested in the second row.

\[
\hat{\beta} = \frac{1}{n \sum_{i=1}^{n}z_i d_i - \sum_{i=1}^{n}d_i \sum_{i=1}^{n}z_i}\sum_{i=1}^{n}-z_i\sum_{i=1}^{n}(y_{i1} - y_{i0}) + n\sum_{i=1}^{n}z_i(y_{i1} - y_{i0})
\]

\[
= \frac{n\sum_{i=1}^{n}z_i(y_{i1} - y_{i0}) - \sum_{i=1}^{n}z_i\sum_{i=1}^{n}(y_{i1} - y_{i0})}{n \sum_{i=1}^{n}z_i d_i - \sum_{i=1}^{n}d_i \sum_{i=1}^{n}z_i}
\]
\\

Dividing the numerator and denominator by \(n\sum_{i=1}^{n}z_i\) yields

\[
\frac{\frac{1}{\sum z_i}\sum_{i=1}^{n}z_i(y_{i1} - y_{i0}) - \frac{1}{n}\sum_{i=1}^{n}(y_{i1} - y_{i0})}{\frac{1}{\sum z_i} \sum_{i=1}^{n}z_i d_i - \frac{1}{n}\sum_{i=1}^{n}d_i}
\]

\noindent\fbox{
    \parbox{\textwidth}{
\[
= (\Delta^1 - \Delta)/(\bar{d}^1 - \bar{d})
\]
}
}
  
\end{homeworkProblem}

\end{document}
