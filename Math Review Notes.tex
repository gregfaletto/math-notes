\documentclass{book}

\usepackage{fancyhdr}
\usepackage{extramarks}
\usepackage{amsmath}
\usepackage{amsthm}
\usepackage{amsfonts}
\usepackage{mathrsfs}
\usepackage{tikz}
\usepackage{enumerate}
\usepackage{graphicx}
\graphicspath{ {images/} }
\usepackage[plain]{algorithm}
\usepackage{algpseudocode}
\usepackage[document]{ragged2e}
\usepackage{textcomp}
\usepackage{color}   %May be necessary if you want to color links
\usepackage{import}
\usepackage{hyperref}
\hypersetup{
    colorlinks=true, %set true if you want colored links
    linktoc=all,     %set to all if you want both sections and subsections linked
    linkcolor=black,  %choose some color if you want links to stand out
}
\usepackage{import}
\usepackage{natbib}
\usepackage{bbm}
\usepackage{bm}
\usepackage{amssymb}
\usepackage{esint}

\usetikzlibrary{automata,positioning}

%% Uncomment this to omit proofs
%\usepackage{environ}
%\NewEnviron{killcontents}{}
%\let\proof\killcontents
%\let\endproof\endkillcontents



%%%%%%
%%%%%% Basic Document Settings
%%%%%%

\topmargin=-0.45in
\evensidemargin=0in
\oddsidemargin=0in
\textwidth=6.5in
\textheight=9.0in
\headsep=0.25in
\setlength{\parskip}{1em}

\linespread{1.1}

\pagestyle{fancy}
\lhead{\hmwkAuthorName}
\lfoot{\lastxmark}
\cfoot{\thepage}

\renewcommand\headrulewidth{0.4pt}
\renewcommand\footrulewidth{0.4pt}

\setlength\parindent{0pt}


\newcommand{\hmwkTitle}{Math Review Notes}
\newcommand{\hmwkAuthorName}{\textbf{G. Faletto} }

%%%%%%
%%%%%% Title Page
%%%%%%

\title{
    \vspace{2in}
    \textmd{\textbf{ \hmwkTitle}}\\
}

\author{Gregory Faletto}
\date{}

\renewcommand{\part}[1]{\textbf{\large Part \Alph{partCounter}}\stepcounter{partCounter}\\}

%%%%%%
%%%%%% Various Helper Commands
%%%%%%

%%%%%% Useful for algorithms
\newcommand{\alg}[1]{\textsc{\bfseries \footnotesize #1}}

%%%%%% For derivatives
\newcommand{\deriv}[2]{\frac{\mathrm{d} #1}{\mathrm{d} #2}}

%%%%%% For partial derivatives
\newcommand{\pderiv}[2]{\frac{\partial #1}{\partial #2}}

%%%%%% Integral dx
\newcommand{\dx}{\mathrm{d}x}

%%%%%% Alias for the Solution section header
\newcommand{\solution}{\textbf{Solution.}}

%%%%%% Probability commands: Expectation, Variance, Covariance, Bias
\newcommand{\E}{\mathbb{E}}
\newcommand{\Var}{\mathrm{Var}}
\newcommand{\Cov}{\mathrm{Cov}}
\newcommand{\Bias}{\mathrm{Bias}}
\newcommand\indep{\protect\mathpalette{\protect\independenT}{\perp}}
\def\independenT#1#2{\mathrel{\rlap{$#1#2$}\mkern2mu{#1#2}}}
\DeclareMathOperator{\Tr}{Tr}
\DeclareMathOperator*{\argmax}{arg\,max}
\DeclareMathOperator*{\argmin}{arg\,min}

\numberwithin{equation}{chapter}

\theoremstyle{definition}
\newtheorem{theorem}{Theorem}
\numberwithin{theorem}{section}
\theoremstyle{definition}
\newtheorem{proposition}[theorem]{Proposition}
%\numberwithin{theorem}{subsection}
\theoremstyle{definition}
\newtheorem{lemma}[theorem]{Lemma}
%\numberwithin{lemma}{subsection}
\theoremstyle{definition}
\newtheorem{corollary}{Corollary}[theorem]
%\numberwithin{corollary}{subsection}
\theoremstyle{definition}
\newtheorem{definition}{Definition}[chapter]
%\numberwithin{definition}{subsection}
\newtheorem{remark}{Remark}
\theoremstyle{definition}
\newtheorem{exercise}{Exercise}
\theoremstyle{definition}
\newtheorem{example}{Example}[chapter]

%%%%%% Tilde
\newcommand{\textapprox}{\raisebox{0.5ex}{\texttildelow}}

\begin{document}

\maketitle

%\pagebreak

\tableofcontents

\


\begin{center}
Last updated \today
\end{center}



%\newpage

% Introduction
\chapter{Introduction}

These are notes I've collected on various math topics. I originally created this document to prepare for the GRE Math Subject test. Since then I've expanded it as I've reviewed concepts from past classes and reinforced concepts from new classes. This document is very much a work in progress, with many typos, omissions to be filled in, and probably errors. Nonetheless, I share this document in case it's useful to anyone else as a reference.

\

I use many sources throughout this document, which I either cite at the beginning of the section (for sources I use broadly) or as I use them (for sources I use for one or two isolated results).

\pagebreak

% Linear Algebra
\import{Topics/Linear_Algebra/}{Linear_Algebra}

\pagebreak

% Calculus
\import{Topics/Calculus/}{Calculus}

\pagebreak

% Differential Equations
\import{Topics/}{Diff_EQ}

\pagebreak

% Real Analysis
\import{Topics/Real_Analysis/}{Real_Analysis}

\pagebreak

% Probability
\import{Topics/Probability/}{Probability}

\pagebreak

% Stochastic Processes
\import{Topics/Stochastic_Processes/}{Stochastic_Processes}

\pagebreak

% Asymptotics
\import{Topics/Asymptotics/}{Asymptotics}

\pagebreak

% Convex Optimization
\import{Topics/Convex_Optimization/}{Convex_Optimization}

\pagebreak

% Mathematical Statistics
\import{Topics/Mathematical_Statistics/}{Mathematical_Statistics}

\pagebreak

% Linear Regression
\import{Topics/LinReg/}{LinReg}

\pagebreak

% Econometrics and Causal Inference
\import{Topics/Econometrics/}{Econometrics}

\pagebreak

% Time Series
\import{Topics/Time_Series/}{Time_Series}

\pagebreak

% Statistical Learning

\import{Topics/Statistical_Learning/}{StatLearning}

\pagebreak

% Abstract Algebra
\import{Topics/Abstract_Algebra/}{Abstract_Algebra}

\pagebreak

% Miscellaneous 
\import{Topics/Miscellaneous/}{Miscellaneous}

\bibliographystyle{abbrvnat}
\bibliography{mybib2fin}

\end{document}